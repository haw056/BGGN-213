\documentclass[]{article}
\usepackage{lmodern}
\usepackage{amssymb,amsmath}
\usepackage{ifxetex,ifluatex}
\usepackage{fixltx2e} % provides \textsubscript
\ifnum 0\ifxetex 1\fi\ifluatex 1\fi=0 % if pdftex
  \usepackage[T1]{fontenc}
  \usepackage[utf8]{inputenc}
\else % if luatex or xelatex
  \ifxetex
    \usepackage{mathspec}
  \else
    \usepackage{fontspec}
  \fi
  \defaultfontfeatures{Ligatures=TeX,Scale=MatchLowercase}
\fi
% use upquote if available, for straight quotes in verbatim environments
\IfFileExists{upquote.sty}{\usepackage{upquote}}{}
% use microtype if available
\IfFileExists{microtype.sty}{%
\usepackage{microtype}
\UseMicrotypeSet[protrusion]{basicmath} % disable protrusion for tt fonts
}{}
\usepackage[margin=1in]{geometry}
\usepackage{hyperref}
\hypersetup{unicode=true,
            pdftitle={Lecture\_8},
            pdfauthor={Howard Wang},
            pdfborder={0 0 0},
            breaklinks=true}
\urlstyle{same}  % don't use monospace font for urls
\usepackage{color}
\usepackage{fancyvrb}
\newcommand{\VerbBar}{|}
\newcommand{\VERB}{\Verb[commandchars=\\\{\}]}
\DefineVerbatimEnvironment{Highlighting}{Verbatim}{commandchars=\\\{\}}
% Add ',fontsize=\small' for more characters per line
\usepackage{framed}
\definecolor{shadecolor}{RGB}{248,248,248}
\newenvironment{Shaded}{\begin{snugshade}}{\end{snugshade}}
\newcommand{\AlertTok}[1]{\textcolor[rgb]{0.94,0.16,0.16}{#1}}
\newcommand{\AnnotationTok}[1]{\textcolor[rgb]{0.56,0.35,0.01}{\textbf{\textit{#1}}}}
\newcommand{\AttributeTok}[1]{\textcolor[rgb]{0.77,0.63,0.00}{#1}}
\newcommand{\BaseNTok}[1]{\textcolor[rgb]{0.00,0.00,0.81}{#1}}
\newcommand{\BuiltInTok}[1]{#1}
\newcommand{\CharTok}[1]{\textcolor[rgb]{0.31,0.60,0.02}{#1}}
\newcommand{\CommentTok}[1]{\textcolor[rgb]{0.56,0.35,0.01}{\textit{#1}}}
\newcommand{\CommentVarTok}[1]{\textcolor[rgb]{0.56,0.35,0.01}{\textbf{\textit{#1}}}}
\newcommand{\ConstantTok}[1]{\textcolor[rgb]{0.00,0.00,0.00}{#1}}
\newcommand{\ControlFlowTok}[1]{\textcolor[rgb]{0.13,0.29,0.53}{\textbf{#1}}}
\newcommand{\DataTypeTok}[1]{\textcolor[rgb]{0.13,0.29,0.53}{#1}}
\newcommand{\DecValTok}[1]{\textcolor[rgb]{0.00,0.00,0.81}{#1}}
\newcommand{\DocumentationTok}[1]{\textcolor[rgb]{0.56,0.35,0.01}{\textbf{\textit{#1}}}}
\newcommand{\ErrorTok}[1]{\textcolor[rgb]{0.64,0.00,0.00}{\textbf{#1}}}
\newcommand{\ExtensionTok}[1]{#1}
\newcommand{\FloatTok}[1]{\textcolor[rgb]{0.00,0.00,0.81}{#1}}
\newcommand{\FunctionTok}[1]{\textcolor[rgb]{0.00,0.00,0.00}{#1}}
\newcommand{\ImportTok}[1]{#1}
\newcommand{\InformationTok}[1]{\textcolor[rgb]{0.56,0.35,0.01}{\textbf{\textit{#1}}}}
\newcommand{\KeywordTok}[1]{\textcolor[rgb]{0.13,0.29,0.53}{\textbf{#1}}}
\newcommand{\NormalTok}[1]{#1}
\newcommand{\OperatorTok}[1]{\textcolor[rgb]{0.81,0.36,0.00}{\textbf{#1}}}
\newcommand{\OtherTok}[1]{\textcolor[rgb]{0.56,0.35,0.01}{#1}}
\newcommand{\PreprocessorTok}[1]{\textcolor[rgb]{0.56,0.35,0.01}{\textit{#1}}}
\newcommand{\RegionMarkerTok}[1]{#1}
\newcommand{\SpecialCharTok}[1]{\textcolor[rgb]{0.00,0.00,0.00}{#1}}
\newcommand{\SpecialStringTok}[1]{\textcolor[rgb]{0.31,0.60,0.02}{#1}}
\newcommand{\StringTok}[1]{\textcolor[rgb]{0.31,0.60,0.02}{#1}}
\newcommand{\VariableTok}[1]{\textcolor[rgb]{0.00,0.00,0.00}{#1}}
\newcommand{\VerbatimStringTok}[1]{\textcolor[rgb]{0.31,0.60,0.02}{#1}}
\newcommand{\WarningTok}[1]{\textcolor[rgb]{0.56,0.35,0.01}{\textbf{\textit{#1}}}}
\usepackage{graphicx,grffile}
\makeatletter
\def\maxwidth{\ifdim\Gin@nat@width>\linewidth\linewidth\else\Gin@nat@width\fi}
\def\maxheight{\ifdim\Gin@nat@height>\textheight\textheight\else\Gin@nat@height\fi}
\makeatother
% Scale images if necessary, so that they will not overflow the page
% margins by default, and it is still possible to overwrite the defaults
% using explicit options in \includegraphics[width, height, ...]{}
\setkeys{Gin}{width=\maxwidth,height=\maxheight,keepaspectratio}
\IfFileExists{parskip.sty}{%
\usepackage{parskip}
}{% else
\setlength{\parindent}{0pt}
\setlength{\parskip}{6pt plus 2pt minus 1pt}
}
\setlength{\emergencystretch}{3em}  % prevent overfull lines
\providecommand{\tightlist}{%
  \setlength{\itemsep}{0pt}\setlength{\parskip}{0pt}}
\setcounter{secnumdepth}{0}
% Redefines (sub)paragraphs to behave more like sections
\ifx\paragraph\undefined\else
\let\oldparagraph\paragraph
\renewcommand{\paragraph}[1]{\oldparagraph{#1}\mbox{}}
\fi
\ifx\subparagraph\undefined\else
\let\oldsubparagraph\subparagraph
\renewcommand{\subparagraph}[1]{\oldsubparagraph{#1}\mbox{}}
\fi

%%% Use protect on footnotes to avoid problems with footnotes in titles
\let\rmarkdownfootnote\footnote%
\def\footnote{\protect\rmarkdownfootnote}

%%% Change title format to be more compact
\usepackage{titling}

% Create subtitle command for use in maketitle
\providecommand{\subtitle}[1]{
  \posttitle{
    \begin{center}\large#1\end{center}
    }
}

\setlength{\droptitle}{-2em}

  \title{Lecture\_8}
    \pretitle{\vspace{\droptitle}\centering\huge}
  \posttitle{\par}
    \author{Howard Wang}
    \preauthor{\centering\large\emph}
  \postauthor{\par}
      \predate{\centering\large\emph}
  \postdate{\par}
    \date{10/25/2019}


\begin{document}
\maketitle

\#k-menas

\begin{Shaded}
\begin{Highlighting}[]
\NormalTok{tmp <-}\StringTok{ }\KeywordTok{c}\NormalTok{(}\KeywordTok{rnorm}\NormalTok{(}\DecValTok{30}\NormalTok{,}\OperatorTok{-}\DecValTok{3}\NormalTok{), }\KeywordTok{rnorm}\NormalTok{(}\DecValTok{30}\NormalTok{,}\DecValTok{3}\NormalTok{)) }
\NormalTok{x <-}\StringTok{ }\KeywordTok{cbind}\NormalTok{(}\DataTypeTok{x=}\NormalTok{tmp, }\DataTypeTok{y=}\KeywordTok{rev}\NormalTok{(tmp))}

\KeywordTok{plot}\NormalTok{(x)}
\end{Highlighting}
\end{Shaded}

\includegraphics{Lecture_8_files/figure-latex/unnamed-chunk-1-1.pdf}

\begin{Shaded}
\begin{Highlighting}[]
\NormalTok{k <-}\StringTok{ }\KeywordTok{kmeans}\NormalTok{(x, }\DataTypeTok{centers =} \DecValTok{2}\NormalTok{, }\DataTypeTok{nstart =} \DecValTok{20}\NormalTok{)}
\end{Highlighting}
\end{Shaded}

\begin{Shaded}
\begin{Highlighting}[]
\KeywordTok{plot}\NormalTok{(x, }\DataTypeTok{col =}\NormalTok{ k}\OperatorTok{$}\NormalTok{cluster)}
\KeywordTok{points}\NormalTok{(k}\OperatorTok{$}\NormalTok{centers, }\DataTypeTok{col =} \StringTok{"blue"}\NormalTok{)}
\end{Highlighting}
\end{Shaded}

\includegraphics{Lecture_8_files/figure-latex/unnamed-chunk-3-1.pdf}

\hypertarget{hierarchical-clustering}{%
\subsection{Hierarchical clustering}\label{hierarchical-clustering}}

\begin{Shaded}
\begin{Highlighting}[]
\NormalTok{hc <-}\StringTok{ }\KeywordTok{hclust}\NormalTok{ (}\KeywordTok{dist}\NormalTok{(x))}
\end{Highlighting}
\end{Shaded}

\begin{Shaded}
\begin{Highlighting}[]
\KeywordTok{plot}\NormalTok{(hc)}
\KeywordTok{abline}\NormalTok{(}\DataTypeTok{h =} \DecValTok{6}\NormalTok{, }\DataTypeTok{col =} \StringTok{"red"}\NormalTok{)}
\end{Highlighting}
\end{Shaded}

\includegraphics{Lecture_8_files/figure-latex/unnamed-chunk-5-1.pdf}

\begin{Shaded}
\begin{Highlighting}[]
\NormalTok{grps <-}\StringTok{ }\KeywordTok{cutree}\NormalTok{(hc, }\DataTypeTok{h =} \DecValTok{4}\NormalTok{)}
\KeywordTok{plot}\NormalTok{ (x, }\DataTypeTok{col =}\NormalTok{ grps)}
\end{Highlighting}
\end{Shaded}

\includegraphics{Lecture_8_files/figure-latex/unnamed-chunk-5-2.pdf}

\begin{Shaded}
\begin{Highlighting}[]
\NormalTok{x <-}\StringTok{ }\KeywordTok{rbind}\NormalTok{(}
\KeywordTok{matrix}\NormalTok{(}\KeywordTok{rnorm}\NormalTok{(}\DecValTok{100}\NormalTok{, }\DataTypeTok{mean=}\DecValTok{0}\NormalTok{, }\DataTypeTok{sd =} \FloatTok{0.3}\NormalTok{), }\DataTypeTok{ncol =} \DecValTok{2}\NormalTok{), }
\KeywordTok{matrix}\NormalTok{(}\KeywordTok{rnorm}\NormalTok{(}\DecValTok{100}\NormalTok{, }\DataTypeTok{mean =} \DecValTok{1}\NormalTok{, }\DataTypeTok{sd =} \FloatTok{0.3}\NormalTok{), }\DataTypeTok{ncol =} \DecValTok{2}\NormalTok{), }
\KeywordTok{matrix}\NormalTok{(}\KeywordTok{c}\NormalTok{(}\KeywordTok{rnorm}\NormalTok{(}\DecValTok{50}\NormalTok{, }\DataTypeTok{mean =} \DecValTok{1}\NormalTok{, }\DataTypeTok{sd =} \FloatTok{0.3}\NormalTok{), }
        \KeywordTok{rnorm}\NormalTok{(}\DecValTok{50}\NormalTok{, }\DataTypeTok{mean =} \DecValTok{0}\NormalTok{, }\DataTypeTok{sd =} \FloatTok{0.3}\NormalTok{)), }\DataTypeTok{ncol =} \DecValTok{2}\NormalTok{))}
\KeywordTok{colnames}\NormalTok{(x) <-}\StringTok{ }\KeywordTok{c}\NormalTok{(}\StringTok{"x"}\NormalTok{, }\StringTok{"y"}\NormalTok{)}
\KeywordTok{plot}\NormalTok{(x)}
\end{Highlighting}
\end{Shaded}

\includegraphics{Lecture_8_files/figure-latex/unnamed-chunk-6-1.pdf}

\begin{Shaded}
\begin{Highlighting}[]
\NormalTok{col <-}\StringTok{ }\KeywordTok{as.factor}\NormalTok{( }\KeywordTok{rep}\NormalTok{(}\KeywordTok{c}\NormalTok{(}\StringTok{"c1"}\NormalTok{,}\StringTok{"c2"}\NormalTok{,}\StringTok{"c3"}\NormalTok{), }\DataTypeTok{each=}\DecValTok{50}\NormalTok{) )}

\KeywordTok{plot}\NormalTok{(x, }\DataTypeTok{col=}\NormalTok{col)}
\end{Highlighting}
\end{Shaded}

\includegraphics{Lecture_8_files/figure-latex/unnamed-chunk-6-2.pdf}

\begin{Shaded}
\begin{Highlighting}[]
\NormalTok{x <-}\StringTok{ }\KeywordTok{rbind}\NormalTok{(}
\KeywordTok{matrix}\NormalTok{(}\KeywordTok{rnorm}\NormalTok{(}\DecValTok{100}\NormalTok{, }\DataTypeTok{mean=}\DecValTok{0}\NormalTok{, }\DataTypeTok{sd =} \FloatTok{0.3}\NormalTok{), }\DataTypeTok{ncol =} \DecValTok{2}\NormalTok{), }
\KeywordTok{matrix}\NormalTok{(}\KeywordTok{rnorm}\NormalTok{(}\DecValTok{100}\NormalTok{, }\DataTypeTok{mean =} \DecValTok{1}\NormalTok{, }\DataTypeTok{sd =} \FloatTok{0.3}\NormalTok{), }\DataTypeTok{ncol =} \DecValTok{2}\NormalTok{), }
\KeywordTok{matrix}\NormalTok{(}\KeywordTok{c}\NormalTok{(}\KeywordTok{rnorm}\NormalTok{(}\DecValTok{50}\NormalTok{, }\DataTypeTok{mean =} \DecValTok{1}\NormalTok{, }\DataTypeTok{sd =} \FloatTok{0.3}\NormalTok{), }
        \KeywordTok{rnorm}\NormalTok{(}\DecValTok{50}\NormalTok{, }\DataTypeTok{mean =} \DecValTok{0}\NormalTok{, }\DataTypeTok{sd =} \FloatTok{0.3}\NormalTok{)), }\DataTypeTok{ncol =} \DecValTok{2}\NormalTok{))}
\KeywordTok{colnames}\NormalTok{(x) <-}\StringTok{ }\KeywordTok{c}\NormalTok{(}\StringTok{"x"}\NormalTok{, }\StringTok{"y"}\NormalTok{)}
\NormalTok{hc <-}\StringTok{ }\KeywordTok{hclust}\NormalTok{(}\KeywordTok{dist}\NormalTok{(x))}
\KeywordTok{plot}\NormalTok{(hc)}
\end{Highlighting}
\end{Shaded}

\includegraphics{Lecture_8_files/figure-latex/unnamed-chunk-7-1.pdf}

\begin{Shaded}
\begin{Highlighting}[]
\NormalTok{grps <-}\StringTok{ }\KeywordTok{cutree}\NormalTok{(hc, }\DataTypeTok{k =} \DecValTok{3}\NormalTok{)}
\KeywordTok{plot}\NormalTok{(x, }\DataTypeTok{col=}\NormalTok{grps)}
\end{Highlighting}
\end{Shaded}

\includegraphics{Lecture_8_files/figure-latex/unnamed-chunk-7-2.pdf}

\begin{Shaded}
\begin{Highlighting}[]
\NormalTok{mydata <-}\StringTok{ }\KeywordTok{read.csv}\NormalTok{(}\StringTok{"https://tinyurl.com/expression-CSV"}\NormalTok{, }\DataTypeTok{row.names=}\DecValTok{1}\NormalTok{)}
\KeywordTok{head}\NormalTok{(mydata)}
\end{Highlighting}
\end{Shaded}

\begin{verbatim}
##        wt1 wt2  wt3  wt4 wt5 ko1 ko2 ko3 ko4 ko5
## gene1  439 458  408  429 420  90  88  86  90  93
## gene2  219 200  204  210 187 427 423 434 433 426
## gene3 1006 989 1030 1017 973 252 237 238 226 210
## gene4  783 792  829  856 760 849 856 835 885 894
## gene5  181 249  204  244 225 277 305 272 270 279
## gene6  460 502  491  491 493 612 594 577 618 638
\end{verbatim}

\begin{Shaded}
\begin{Highlighting}[]
\NormalTok{pca <-}\StringTok{ }\KeywordTok{prcomp}\NormalTok{(}\KeywordTok{t}\NormalTok{(mydata), }\DataTypeTok{scale=}\OtherTok{TRUE}\NormalTok{)}
\KeywordTok{attributes}\NormalTok{(pca)}
\end{Highlighting}
\end{Shaded}

\begin{verbatim}
## $names
## [1] "sdev"     "rotation" "center"   "scale"    "x"       
## 
## $class
## [1] "prcomp"
\end{verbatim}

\begin{Shaded}
\begin{Highlighting}[]
\KeywordTok{plot}\NormalTok{(pca}\OperatorTok{$}\NormalTok{x[,}\DecValTok{1}\NormalTok{], pca}\OperatorTok{$}\NormalTok{x[,}\DecValTok{2}\NormalTok{])}
\end{Highlighting}
\end{Shaded}

\includegraphics{Lecture_8_files/figure-latex/unnamed-chunk-9-1.pdf}

\begin{Shaded}
\begin{Highlighting}[]
\NormalTok{pca.var <-}\StringTok{ }\NormalTok{pca}\OperatorTok{$}\NormalTok{sdev}\OperatorTok{^}\DecValTok{2} 
\NormalTok{pca.var.per <-}\StringTok{ }\KeywordTok{round}\NormalTok{(pca.var}\OperatorTok{/}\KeywordTok{sum}\NormalTok{(pca.var)}\OperatorTok{*}\DecValTok{100}\NormalTok{, }\DecValTok{1}\NormalTok{)}
\KeywordTok{barplot}\NormalTok{(pca.var.per, }\DataTypeTok{main=}\StringTok{"Scree Plot"}\NormalTok{,}
\DataTypeTok{xlab=}\StringTok{"Principal Component"}\NormalTok{, }\DataTypeTok{ylab=}\StringTok{"Percent Variation"}\NormalTok{)}
\end{Highlighting}
\end{Shaded}

\includegraphics{Lecture_8_files/figure-latex/unnamed-chunk-9-2.pdf}

\begin{Shaded}
\begin{Highlighting}[]
\NormalTok{x <-}\StringTok{ }\KeywordTok{read.csv}\NormalTok{(}\StringTok{"https://bioboot.github.io/bggn213_f17/class-material/UK_foods.csv"}\NormalTok{, }\DataTypeTok{row.names=}\DecValTok{1}\NormalTok{)}
\KeywordTok{barplot}\NormalTok{(}\KeywordTok{as.matrix}\NormalTok{(x), }\DataTypeTok{beside=}\NormalTok{F, }\DataTypeTok{col=}\KeywordTok{rainbow}\NormalTok{(}\KeywordTok{nrow}\NormalTok{(x)))}
\end{Highlighting}
\end{Shaded}

\includegraphics{Lecture_8_files/figure-latex/unnamed-chunk-10-1.pdf}

\begin{Shaded}
\begin{Highlighting}[]
\KeywordTok{pairs}\NormalTok{(x, }\DataTypeTok{col=}\KeywordTok{rainbow}\NormalTok{(}\DecValTok{10}\NormalTok{), }\DataTypeTok{pch=}\DecValTok{16}\NormalTok{)}
\end{Highlighting}
\end{Shaded}

\includegraphics{Lecture_8_files/figure-latex/unnamed-chunk-10-2.pdf}

\begin{Shaded}
\begin{Highlighting}[]
\NormalTok{pca <-}\StringTok{ }\KeywordTok{prcomp}\NormalTok{( }\KeywordTok{t}\NormalTok{(x) )}
\KeywordTok{summary}\NormalTok{(pca)}
\end{Highlighting}
\end{Shaded}

\begin{verbatim}
## Importance of components:
##                             PC1      PC2      PC3       PC4
## Standard deviation     324.1502 212.7478 73.87622 4.189e-14
## Proportion of Variance   0.6744   0.2905  0.03503 0.000e+00
## Cumulative Proportion    0.6744   0.9650  1.00000 1.000e+00
\end{verbatim}

\begin{Shaded}
\begin{Highlighting}[]
\KeywordTok{plot}\NormalTok{(pca}\OperatorTok{$}\NormalTok{x[,}\DecValTok{1}\OperatorTok{:}\DecValTok{2}\NormalTok{], }\DataTypeTok{xlab=}\StringTok{"PC1"}\NormalTok{, }\DataTypeTok{ylab=}\StringTok{"PC2"}\NormalTok{, }\DataTypeTok{xlim=}\KeywordTok{c}\NormalTok{(}\OperatorTok{-}\DecValTok{270}\NormalTok{,}\DecValTok{500}\NormalTok{))}
\KeywordTok{text}\NormalTok{(pca}\OperatorTok{$}\NormalTok{x[,}\DecValTok{1}\NormalTok{], pca}\OperatorTok{$}\NormalTok{x[,}\DecValTok{2}\NormalTok{], }\KeywordTok{colnames}\NormalTok{(x), }\DataTypeTok{col =} \KeywordTok{c}\NormalTok{(}\StringTok{"red"}\NormalTok{, }\StringTok{"blue"}\NormalTok{))}
\end{Highlighting}
\end{Shaded}

\includegraphics{Lecture_8_files/figure-latex/unnamed-chunk-10-3.pdf}


\end{document}
